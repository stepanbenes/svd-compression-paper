%% This is file `elsarticle-template-1-num.tex',
%%
%% Copyright 2009 Elsevier Ltd
%%
%% This file is part of the 'Elsarticle Bundle'.
%% ---------------------------------------------
%%
%% It may be distributed under the conditions of the LaTeX Project Public
%% License, either version 1.2 of this license or (at your option) any
%% later version.  The latest version of this license is in
%%    http://www.latex-project.org/lppl.txt
%% and version 1.2 or later is part of all distributions of LaTeX
%% version 1999/12/01 or later.
%%
%% Template article for Elsevier's document class `elsarticle'
%% with numbered style bibliographic references
%%
%% $Id: elsarticle-template-1-num.tex 149 2009-10-08 05:01:15Z rishi $
%% $URL: http://lenova.river-valley.com/svn/elsbst/trunk/elsarticle-template-1-num.tex $
%%
\documentclass[preprint,12pt]{elsarticle}

%% Use the option review to obtain double line spacing
%\documentclass[preprint,review,12pt]{elsarticle}

%% Use the options 1p,twocolumn; 3p; 3p,twocolumn; 5p; or 5p,twocolumn
%% for a journal layout:
%% \documentclass[final,1p,times]{elsarticle}
%% \documentclass[final,1p,times,twocolumn]{elsarticle}
%% \documentclass[final,3p,times]{elsarticle}
%% \documentclass[final,3p,times,twocolumn]{elsarticle}
%% \documentclass[final,5p,times]{elsarticle}
%% \documentclass[final,5p,times,twocolumn]{elsarticle}

%% The graphicx package provides the includegraphics command.
\usepackage{graphicx}
%% The amssymb package provides various useful mathematical symbols
\usepackage{amssymb}
%% The amsthm package provides extended theorem environments
%% \usepackage{amsthm}
%% following packages allows to write algorithms as pseudocode
\usepackage{amsmath}
\usepackage{algorithmicx}
\usepackage{algorithm}
\usepackage{algpseudocode}
\algnewcommand\algorithmicinput{\textbf{INPUT:}}
\algnewcommand\INPUT{\item[\algorithmicinput]}
\algnewcommand\algorithmicoutput{\textbf{OUTPUT:}}
\algnewcommand\OUTPUT{\item[\algorithmicoutput]}

%% The lineno packages adds line numbers. Start line numbering with
%% \begin{linenumbers}, end it with \end{linenumbers}. Or switch it on
%% for the whole article with \linenumbers after \end{frontmatter}.
\usepackage{lineno}

% math tools (ceiling signs, see: http://tex.stackexchange.com/questions/42271/floor-and-ceiling-functions)
\usepackage{mathtools}
\DeclarePairedDelimiter{\ceil}{\lceil}{\rceil}

% new commands
\newcommand{\mtrx}[1]{\mathbf{#1}}

\journal{Advances in Engineering Software}

\begin{document}

\begin{frontmatter}

%% Title, authors and addresses

\title{Singular Value Decomposition Used for Compression of Results from the Finite Element Method}

\author{\v{S}t\v{e}p\'{a}n Bene\v{s}} % Stepan Benes
\ead{stepan.benes@fsv.cvut.cz}
\author{Jaroslav Kruis}
%\author{Jaroslav Kruis\corref{correspondingauthor}}
%\cortext[correspondingauthor]{Corresponding author. Tel.: + 420 224354580.}
\ead{jaroslav.kruis@fsv.cvut.cz}

\address{Department of Mechanics, Faculty of Civil Engineering, Czech Technical University in Prague Th\'akurova 7, Prague 166 29, Czech Republic}

\begin{abstract}
%% Text of abstract should have three parts: Motivation. This paper. Summary.
A complex finite element analysis can produce large amount of data that is problematic to post-process in reasonable time. This paper describes application of Singular Value Decomposition (SVD) to the compression of results from finite element solvers. Although the idea of image compression method is an inspiration for this research work, the SVD compression algorithm used for compression of images cannot be directly used for FEM results. Differences and implementation challenges are discussed in the text. Quality of approximation is more important in scientific field then in computer graphics where the most significant factor is the human perception of the resulting image. Error estimation methods used during compression of finite element results are presented. The focus is also on the algorithm performance. SVD is a very computational intensive method. Therefore, various optimization techniques were investigated, e.g. randomized SVD. The method leads to the lower memory consumption, 10\% of the original size or less, with negligible compression error.
\end{abstract}

\begin{keyword}
Finite Element Method (FEM) \sep Compression \sep Singular Value Decomposition (SVD) \sep Randomized SVD \sep Data Visualization \sep Data Post-processing
\end{keyword}

\end{frontmatter}

%%
%% Start line numbering here if you want
%%
%\linenumbers

%% main text

% State of the art in FEM data compression
\section{Introduction}
\label{sec:introduction}

% Introduction to field of FEM results post-processing.

The effort to reduce size of resulting data from complex finite element analyses to accelerate (or even make possible) post-processing using common personal computer is not new.
\todo[inline]{State of the art. Add references to work in this field.}
In \cite{Benes2016} is presented one possible direction for research in this area -- the replacement of discrete data produced by finite element solver by simple continuous functions. These approximation functions can be then described be low number of parameters. The focus is on \textit{geometry}. The main goal is to find the areas in domain where the output discrete function has predictable development and can be easily replaced by e.g. linear continuous function. Although this approach has great results for some class of functions it is not applicable to problem in general. For some special cases, such as functions with discontinuities, the method has poor results because approximation error is too high and -- what is more important -- it can not be controlled in advance.

In this study the purely \textit{algebraic} approach is applied.

% TODO: Description of all terms that are used in text

\subsection{Related work / State of the art}

% Todo: Describe alternative methods for compression: e.g. discrete 9/7 bi-orthogonal wavelet transform, discrete cosine transform, Karhunen-Loeve transform, and combinations of these.

% Mathematical background -- SVD, Low-rank approximation matrix, Randomized SVD
\section{Mathematical background}
\label{sec:math}

\todo[inline]{Singular Value Decomposition in general}

% Describe how to calculate decomposition of arbitrary matrix with example

\subsection{Error estimation}
\label{sec:error}

\todo[inline]{Singular values and norms of matrix}

(following taken from \cite{SairaBanu2015})

Mean square error:
\begin{equation}
MSE=\frac{1}{m n} \sum_{i=1}^{m} \sum_{j=1}^{n} (A_{ij} - A'_{ij})^{2}
\label{eq:mse-def}
\end{equation}

where \textit{A} represents the original image and \textit{A'} represents the
reconstructed image of dimension $m \times n$.

Rooted Mean Square Deviation:
\begin{equation}
RMSD=\sqrt{MSE}
\label{eq:rmsd-def}
\end{equation}

Normalized Rooted Mean Square Deviation:
\begin{equation}
NRMSD=\frac{RMSD}{X_{max}-X_{min}}=\frac{\sqrt{MSE}}{X_{max}-X_{min}}
\label{eq:nrmsd-def}
\end{equation}

Peak signal to noise ratio [db]:
\begin{equation}
PSNR=10\log_{10}\frac{(X_{max}-X_{min})^{2}}{MSE}
\end{equation}

\begin{equation}
PSNR=20\log_{10}\frac{X_{max}-X_{min}}{\sqrt{MSE}}=20\log_{10}\frac{1}{NRMSD}
\end{equation}

\begin{equation}
PSNR=-20\log_{10}NRMSD
\label{eq:psnr-def}
\end{equation}

\subsection{Sparse matrix of details}

\todo[inline]{Mozna tento odstavec uplne vyhodit, stejne to neni implementovano -- uvest jen v sekci Optimization nebo v Conclusion}

\subsection{Randomized SVD}
Exact SVD of a $m \times n$ matrix has time complexity $O(min(mn^2, m^2n))$ using the "big-O" notation.

Assume the data points are in the columns of $\mtrx{A} \in \mtrx{M_{m,n}}(\mathbb{R})$ where $m \leq n$. Note that $\mtrx{AA^T}$ is the dataset covariance matrix. Then a simple method is to randomly choose $k<m$ columns of $\mtrx{A}$ that form a matrix $\mtrx{S}$. Statistically, the SVD of $\mtrx{SS^T}$ will be close to that of $\mtrx{AA^T}$; thus it suffices to calculate the SVD of $\mtrx{S}$, the complexity of which, is only $O(k^2m)$.

\todo[inline]{Je tohle stejny randomizovany algoritmus jako pouzivam ja v RedSVD? V clanku se mluvi o Monte Carlo...}

% http://mathoverflow.net/questions/161252/what-is-the-time-complexity-of-truncated-svd
% http://sysrun.haifa.il.ibm.com/hrl/bigml/files/Holmes.pdf

% Description of FEM data compression algorithm
\section{Implementation}
\label{sec:implementation}

% SVD is applied on matrices. The first thing to do is therefore to assemble an input matrix. ...
% As an example, temperature field, vector of nodal displacements, strain tensor evaluated in integration points, etc. can serve. There are two similar sets of results. One is generated by a non-linear algorithms, where several incremental steps are stored and the other is generated by time integration, where results in particular time steps are stored. ...

Results from the finite element method are scalar, vector or tensor fields represented by discrete values calculated in nodes of the mesh or in integration points on finite elements. In order to compress data, an auxiliary matrix~$\mtrx{A}$ has to be assembled from the results. The number of rows of the matrix~$\mtrx{A}$ is equal to the number of incremental or time steps while the number of columns is equal to the number of points in which the results are stored. Such auxiliary matrix is assembled for each scalar field and for each component of the vector and tensor fields. It means, three matrices corresponding to the displacement in the $x$, $y$, and $z$ directions are assembled for the vector of displacements in three-dimensional problems.

There are two main reasons to store particular results in separate matrices. First, the size of matrices is smaller than the size of a matrix which contains all results and therefore SVD will be performed faster. Second, the magnitudes of particular fields are very different (the stress tensor components are several order of magnitude larger than the components of the displacement vector) and the data compression algorithm would suppress the fields with small magnitudes. Once the matrix $\mtrx{A}$ is assembled for each field, the compression algorithm can be applied on it. It is purely algebraic procedure and no information about geometry of the mesh is needed.

Let us assume that the matrix is not empty and is full rank. Then it follows from the formula (\ref{eq:cr-def}) that if $r$ is equal to the rank of matrix $\mtrx{A}$, the compression ratio is always higher than one. In other words the memory consumption of stored decomposition is bigger than the size of the original matrix. To make the compression algorithm applicable, the parameter $r$ must satisfy the condition

\begin{equation}
r<\frac{m n}{m+n+1}.
\label{eq:r-ineq}
\end{equation}

\noindent
Considering the usual shape of matrix containing FEM results, this inequality is easily satisfiable even for the $r$ being close to the rank of the original matrix as in the typical case the number of nodes or integration points is much higher than the number of analysis steps and therefore $m \ll n$.

\subsection{Algorithm description}
Once SVD is calculated, the compression algorithm removes a certain number of singular values and corresponding singular vectors. The remaining singular values and vectors represent the compressed data. There are two strategies that influence the way how to preserve the number of singular values -- resulting size and quality. Each strategy is assigned a control parameter that determines compression ratio or approximation error.

\paragraph{Compression ratio}
If the focus is only on the size of compressed data, the rank $r$ of the approximation matrix can be calculated by the formula

\begin{equation}
r=\ceil*{c \times \frac{m n}{m+n+1}},
\label{eq:rank-from-comp-ratio}
\end{equation}

\noindent
where $c$ is the compression ratio, $0 \leq c \leq 1$ ($0$ results in absolute compression while $1$ results in no compression); $\ceil*{.}$ is the ceiling function.

\paragraph{Approximation error}
In a usual case, the most important measure to take into account is the approximation error. Algorithm is trying to minimize the compression ratio while at the same time ensuring that predefined approximation error threshold is not exceeded. To quantify the error, the Normalized root-mean-square deviation ($\mathit{NRMSD}$) is used. The normalized error metric enables working with various data sets that have different scales. $\mathit{NRMSD}$ is defined in Section \ref{sec:error}.

To effectively calculate the final rank of the approximation matrix from the desired approximation error, the interesting property of singular values

\begin{equation}
\sum_{i=1}^{m} \sum_{j=1}^{n} (a_{ij})^{2} = \sum_{i=1}^{k}{s_{i}^{2}},
\label{eq:elem-sqr-sigma-sqr}
\end{equation}

\noindent
where $k=\mathrm{min}(m, n)$, i.e. the smallest of two dimensions of the matrix $\mtrx{A}$, is made use of. The above formula states that the sum of squared elements of the matrix $\mtrx{A}$ equals to the sum of squared singular values $s_{i}$ of the same matrix $\mtrx{A}$.

Using formulas \eqref{eq:svd-expansion} and \eqref{eq:svd-approx-expansion} the equation \eqref{eq:elem-sqr-sigma-sqr} can be applied to the difference between original matrix $\mtrx{A}$ and approximation matrix $\mtrx{A'}$

\begin{equation}
\sum_{i=1}^{m} \sum_{j=1}^{n} (a_{ij} - a'_{ij})^{2} = \sum_{i=r+1}^{k}{s_{i}^{2}},
\end{equation}

\noindent
where the term on the right-hand side is the sum of squares of those singular values of the matrix $\mtrx{A}$ that are going to be cut away by the compression algorithm. The equation can be rewritten using the definition of $\mathit{MSE}$ in \eqref{eq:mse-def} to

\begin{equation}
\mathit{MSE} \times m n = \sum_{i=r+1}^{k} s_{i}^{2}
\end{equation}

\noindent
and using \eqref{eq:nrmsd-def} further to

\begin{equation}
(\mathit{NRMSD} \times (X_{max}-X_{min}))^{2} \times m n = \sum_{i=r+1}^{k} s_{i}^{2}.
\end{equation}

Then $\mathit{NRMSD}$ can be used as a quality metric for the compression algorithm because normalization makes it usable for different datasets. Calculation of rank of the approximation matrix is depicted as pseudo-code in Algorithm \ref{alg:rank-calculation}. Algorithm uses the inequality

\begin{equation}
e > \frac{\sqrt[]{\frac{\sum_{i=r+1}^{k} s_{i}^{2}}{m n}}}{X_{max}-X_{min}}
\end{equation}

\noindent
to test whether the desired rank has been reached; $e$ is $\mathit{NRMSD}$ used as an error threshold that can not be exceeded to achieve desired quality of approximation.

\begin{algorithm}
  \caption{Calculation of rank for approximation matrix from maximum allowed error}\label{rankAlgorithm}
  \label{alg:rank-calculation}
  \begin{algorithmic}[1]
  	\INPUT maximum allowed error ($e: e > 0$), array with singular values ($S: S.length > 0$), element count ($c: c > 0$), maximum element value ($x_{max}$), minimum element value ($x_{min}: x_{max} > x_{min}$)
    \OUTPUT rank of resulting matrix
    \Procedure{CalculateRank}{$e, S, c, x_{max}, x_{min}$}
      \State $\mathit{MSE} \gets 0$
      \State $\mathit{NRMSD} \gets 0$
      \State $rank \gets S.length$
      \While{$\mathit{NRMSD} < e$}\Comment{repeat until max error is reached}
        \State $\mathit{MSE} \gets \mathit{MSE} + S[rank]/c$ \Comment{calculate $\mathit{MSE}$ for current rank}
        \State $\mathit{NRMSD} \gets \sqrt{\mathit{MSE}} / (x_{max} - x_{min})$ \Comment{normalize error}
        \State $rank \gets rank - 1$ \Comment{decrement rank for next loop}
      \EndWhile
      \State \textbf{return} $rank + 1$ \Comment{Add one to not exceed maximum allowed error}
    \EndProcedure
  \end{algorithmic}
\end{algorithm}

\subsection{Optimization}

Computational complexity of the exact SVD algorithm is $\mathrm{O}(m^2n)$, where $m<n$. This theoretical algorithm complexity is confirmed by two benchmarks where the dependency of the execution time on the varying matrix dimension is shown. The results of the benchmarks are depicted in Figure \ref{fig:ExeTime_rows} and Figure \ref{fig:ExeTime_columns}. Several observations were made from the results:

\begin{itemize}
\item The algorithm is most efficient in cases where one dimension of the input matrix is very small compared to the other. However, this is almost always the case when compressing results from FEM -- number of incremental or time steps seldom exceeds hundreds.
\item Moreover, incremental or time steps can be devided into smaller ranges and the algorithm can be applied on each range separately. This will improve performance and can also increase quality of compression if the key time steps on the range boundaries are carefully selected.
\item The randomized SVD algorithm has the same order of algorithmic complexity when full decomposition is required, but yet can significantly reduce execution time. However, the benchmarks are not designed to highlight the benefits of randomized SVD algorithms. The main advantage of the randomized SVD is in the ability to choose the rank of the approximation matrix in advance. In that case only limited number of singular values and corresponding singular vectors are calculated and algorithm performs much faster.
\end{itemize}

Storage size of SVD itself can also be optimized. $\mtrx{S}$, being a diagonal matrix, can be stored as single list of singular values $s_{i}$, or can be even multiplied with the matrix of left singular vectors $\mtrx{U}$.


% Results of compression method (images from postprocessor, tables with compression ratio and error, etc.)
\section{Results}
\label{section:results}

\begin{figure}[ht]
\centering\includegraphics[width=\textwidth]{figures/temelin_screenshot}
\caption{Figure caption}
\end{figure}

\begin{figure}[ht]
\centering\includegraphics[width=\textwidth]{figures/chotkova_screenshot}
\caption{Figure caption}
\end{figure}

\begin{figure}[ht]
\centering\includegraphics[width=\textwidth]{figures/mechaxisym_screenshot}
\caption{Figure caption}
\end{figure}

\begin{figure}[ht]
\centering\includegraphics[width=\textwidth]{figures/chotkova_NRMSD}
\caption{Figure caption}
\end{figure}

\begin{figure}[ht]
\centering\includegraphics[width=\textwidth]{figures/chotkova_PSNR}
\caption{Figure caption}
\end{figure}

\begin{figure}[ht]
\centering\includegraphics[width=\textwidth]{figures/chotkova_MaxError}
\caption{Figure caption}
\end{figure}

\begin{figure}[ht]
\centering\includegraphics[width=\textwidth]{figures/mechaxisym_NRMSD}
\caption{Figure caption}
\end{figure}

\begin{figure}[ht]
\centering\includegraphics[width=\textwidth]{figures/mechaxisym_PSNR}
\caption{Figure caption}
\end{figure}

\begin{figure}[ht]
\centering\includegraphics[width=\textwidth]{figures/temelin_NRMSD}
\caption{Figure caption}
\end{figure}

\begin{figure}[ht]
\centering\includegraphics[width=\textwidth]{figures/temelin_PSNR}
\caption{Figure caption}
\end{figure}

\begin{figure}[ht]
\centering\includegraphics[width=\textwidth]{figures/temelin_PSNR_rand}
\caption{Figure caption}
\end{figure}

%\begin{table}[ht]
%\centering
%\begin{tabular}{l l l l}
%\hline
%\textbf{Rank} & \textbf{5} & \textbf{10} & \textbf{15}\\
%\hline
%Compression ratio & 0 & 0 & 0 \\
%Compression time \\
%Reconstruction time \\
%Error \\
%\hline
%\end{tabular}
%\caption{Performance characteristics of SVD on "Example name"}
%\end{table}

% Performance characteristics of Randomized SVD on same analyses

% Graphs: Error/Rank, Speed/Rank

% Evaluation of results
\section{Conclusion}
\label{sec:conclusion}

The goal of this paper is to present the use of SVD decomposition in compression of results from the Finite element method. Theoretical properties of SVD decomposition have been described, implementation details of compression algorithm have been provided, and results of its application on real data have been presented.

The implemented algorithm is able to compress arbitrary data using the low-rank approximation matrices. If the max allowed error was set to 1e-5 then the compression ratio was at least 10\% for all tested results. In many cases compression ratio can be even much lower -- bellow 1\% of the original size. The important property of compression algorithm is the fact that the approximation error can be set in advance and there is a guarantee that it will not be exceeded.

The main disadvantage is the computational complexity of the compression algorithm. SVD decomposition is very time consumptive operation. However, this operation is performed only once after FEM analysis is finished and before the post-processing is started. Also, randomized version of SVD decomposition algorithm is much faster if you don't mind the slight increase of approximation error. % da se urychlit pomoci randomized SVD, pokud dopredu znam kompresni pomer a neni pro mne kriticke neprekrocit maximalni chybu.

Other complication is the necessity to use special format to store compressed data in the form of SVD decomposition and the post-processor must be updated to use this new format. Post-processor must perform matrix multiplication to get the original results. However, in usual case the data for one analysis step are needed at a time. In this case only a partial multiplication of single row is enough.

% TODO: Uvest budouci praci? (v minule praci jsem se soustredil na vyuziti zavislosti v prostorove dimenzi. V teto fazi se soustredim na casovou dimenzi. Dalsi moznosti je tyto dva pristupy zkombinovat. Dalsi pokracovani: vyuzit optimalizace popsane v textu, paralelizovat, pouzit matici detailu)


% Acknowledgement
\section*{Acknowledgement}
\todo[inline]{Add acknowledgement}

%% References with bibTeX database
\section*{References}

\bibliographystyle{model1-num-names}
\bibliography{references}

\end{document}
