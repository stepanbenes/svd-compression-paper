\section{Conclusion}
\label{sec:conclusion}

The goal of this paper is to present the use of SVD decomposition in compression of results from the Finite element method. Implementation details of compression algorithm have been provided, and results of its application on real data have been presented. The implemented algorithm is able to compress arbitrary data using low-rank approximation matrices. If the maximum allowed error was set to $10^{-5}$ then the compression ratio was at least 10\% for all tested results. In many cases compression ratio can be even much lower -- bellow 1\% of the original size. The important property of compression algorithm is the fact that the approximation error can be set in advance and there is a guarantee that it will not be exceeded.

The main disadvantage is the computational complexity of the compression algorithm. SVD decomposition is very time consuming operation. However, this operation is performed only once after FEM analysis is finished and before the post-processing is started. Also, randomized version of the SVD decomposition algorithm is much faster and can be used if the slight increase of approximation error is tolerated.

Other complication is the necessity to use special format to store compressed data in the form of SVD decomposition and the post-processor must be updated to use this new format. Post-processor must perform matrix multiplication to get the original results. However, in usual case the data for one analysis step are needed at a time, i.e. only a partial multiplication of a single row is enough. Detailed description of a post-processor implementation is beyond the scope of this paper.
