\section{Conclusion}
\label{sec:conclusion}

The goal of this paper is to present the use of SVD in compression of results from the Finite Element Method. Implementation details of the compression algorithm have been provided, and the results of its application on real data have been presented. The implemented algorithm is able to compress arbitrary data using low-rank approximation matrices. When the maximum allowed error was set to $10^{-5}$, the compression ratio was at most 10\% for all tested results. In many cases compression ratio can be even better -- bellow 1\% of the original size. The important property of the compression algorithm is the fact that the approximation error can be set in advance and there is a guarantee that it will not be exceeded.

The main disadvantage is the computational complexity of the compression algorithm. SVD is a very time-consuming operation. However, this operation is performed only once after FEM analysis is finished and before the post-processing is started. Also, the randomized version of the decomposition algorithm is much faster and can be used if a slight increase of approximation error is tolerated.

Another complication is the necessity to use a special format to store compressed data in the form of matrix decompositions, and the post-processor must be updated to use this new format. Post-processor must perform matrix multiplication to get the original results. However, in a usual case the data for one analysis step are needed at a time, i.e. multiplication of a single row is enough. Detailed description of the post-processor implementation is beyond the scope of this paper.
