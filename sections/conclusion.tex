\section{Conclusion}
\label{sec:conclusion}

%\todo[inline]{Add conclusion - discussion of results mainly}

The goal of this paper is to present the use of SVD decomposition in compression of results from the Finite element method. Theoretical properties of SVD decomposition have been described, implementation details of compression algorithm have been provided, and results of its application on real data have been presented.


% Popsat hlavni vysledek prace: schopnost kompresovat libovolna fem data s vyuzitim matice s nizsi hodnosti (dulezite rict ze toto je ta hlavni metoda, ktera to vlastne kompresuje). U vetsiny testovanych uloh se podarilo dosahnout minimalne 10x komprese, u nekterych az 100x komprese pri zanedbatelne chybe

% Dulezity poznatek: maximalni chyba se da dopredu nastavit

% Nevyhody: pomerne casove narocna operace, da se urychlit pomoci randomized SVD, pokud dopredu znam kompresni pomer a neni pro mne kriticke neprekrocit maximalni chybu.
% Ale: jde spocitat jednou provzdy po skonceni analyzy pred samotnym postprocessingem (na vypocetnim serveru). Samotny postprocessing uz pak vyzaduje jen vynasobeni matice, coz je rychlejsi - navic - staci mi provest jen nasobeni jednoho radku, protoze v jednom okamziku vetsinou chci data jen pro jeden casovy krok

% Shrnuti a budouci prace. v minule praci jsem se soustredil na vyuziti zavislosti v prostorove dimenzi. V teto fazi se soustredim na casovou dimenzi. Dalsi moznosti je tyto dva pristupy zkombinovat. Dalsi pokracovani: vyuzit optimalizace popsane v textu, paralelizovat, pouzit matici detailu

% zkontrolovat abstrakt zda tam rikam stejne veci a kdyztak ho aktualizovat