\section{Mathematical background}
\label{sec:math}

% TODO: Singular Value Decomposition in general
% TODO: Describe how to calculate decomposition of arbitrary matrix with example
% TODO: add more citations separated by comma

Singular value decomposition (SVD) \cite{Baker2013, Kalman2002} is based on a theorem from linear algebra which says that a rectangular matrix $\mtrx{A} \in \mathbb{R}^{m \times n}$ can be decomposed into the product of three matrices - an orthogonal matrix $\mtrx{U} \in \mathbb{R}^{m \times m}$, a diagonal
matrix $\mtrx{S} \in \mathbb{R}^{m \times n}$, and the transpose of an orthogonal matrix $\mtrx{V} \in \mathbb{R}^{n \times n}$:

\begin{equation}
\mtrx{A} = \mtrx{U} \mtrx{S} \mtrx{V}^\mathsf{T},
\label{eq:svd-def}
\end{equation}

\noindent
where $\mtrx{U^\mathsf{T}U} = \mtrx{I}$, $\mtrx{V^\mathsf{T}V} = \mtrx{I}$; the columns of $\mtrx{U}$ are orthonormal eigenvectors of $\mtrx{AA^\mathsf{T}}$, the columns of $\mtrx{V}$ are orthonormal eigenvectors of $\mtrx{A^\mathsf{T}A}$, and $\mtrx{S}$ (sometimes referred to as $\mtrx{\Sigma}$) is a diagonal matrix containing singular values in descending order which are at the same time the square roots of eigenvalues of $\mtrx{U}$ or $\mtrx{V}$.

SVD can be seen as a method for transforming correlated variables into a set of uncorrelated ones. At the same time, SVD is a method for identifying and ordering the dimensions along which data points show the most variation. Once we have identified where the most variation is, it is possible to find the best approximation of the original data points using fewer dimensions. Hence, SVD can be seen as a method for data reduction/compression.

This is the basic idea behind SVD: taking a high dimensional, highly variable set of data points and reducing it to a lower dimensional space that exposes the substructure of the original data more clearly and orders it from most variation to the least. What makes SVD practical for data compression applications is that variation below a particular threshold can be simply ignored to massively reduce data with assurance that the main relationships of interest have been preserved.

\subsection{Error estimation}
\label{sec:error}

% TODO: Singular values and norms of matrix

The objective of data compression is to represent an input with smaller amount of data. However, the data reconstructed from its compressed representation may or may not be the exact copy of the original image. A compression technique can be lossy or lossless based on the quality of data it restores.

Low-rank approximation matrix method that is described in this paper is a lossy compression technique. Several error metrics are used to control the quality of results \cite{SairaBanu2015}.

\begin{itemize}
\item \textbf{Mean square error}
\begin{equation}
MSE=\frac{1}{m n} \sum_{i=1}^{m} \sum_{j=1}^{n} (A_{ij} - A'_{ij})^{2},
\label{eq:mse-def}
\end{equation}

\noindent
where $A_{ij}$ represents an element of the original matrix and $A'_{ij}$ represents an element of the
reconstructed matrix of dimension $m \times n$.

\item \textbf{Rooted Mean Square Deviation}
\begin{equation}
RMSD=\sqrt{MSE}
\label{eq:rmsd-def}
\end{equation}

\item \textbf{Normalized Rooted Mean Square Deviation}
\begin{equation}
NRMSD=\frac{RMSD}{X_{max}-X_{min}}=\frac{\sqrt{MSE}}{X_{max}-X_{min}},
\label{eq:nrmsd-def}
\end{equation}

\noindent
where $X_{min}$ and $X_{max}$ are elements of input matrix $\mtrx{A}$ with minimum and maximum value, respectively. Being able to compare datasets with different scales this is the main parameter that is used to control quality of compression in the algorithm presented in this paper.

\item \textbf{Peak signal to noise ratio}

PSNR is most commonly used to measure the quality of reconstruction of lossy compression methods (e.g. image compression). The signal in this case is the original data, and the noise is the error introduced by compression. PSNR is an approximation to human perception of reconstruction quality. This metric is not so important in area of FEM analyses where the human perception of visualizations is not as important as the exact mathematical accuracy of approximations. The reason to include PSNR in results is in particular to allow comparison with other image-related compression methods. PSNR is usually expressed in terms of the logarithmic decibel scale (dB).

\begin{equation}
PSNR=10\log_{10}\frac{(X_{max}-X_{min})^{2}}{MSE}
\end{equation}

\begin{equation}
PSNR=20\log_{10}\frac{X_{max}-X_{min}}{\sqrt{MSE}}=20\log_{10}\frac{1}{NRMSD}
\end{equation}

\begin{equation}
PSNR=-20\log_{10}NRMSD
\label{eq:psnr-def}
\end{equation}

\end{itemize}

\subsection{Randomized SVD}
In \cite{Holmes, Halko2011, Candes2009} are described randomized algorithms for constructing approximate matrix factorizations. Their main purpose is to accelerate calculations of matrix decompositions which tend to be very slow for large data sets.

Exact SVD of a $m \times n$ matrix has time complexity $O(min(mn^2, m^2n))$ using the "big-O" notation. Assume the data points are in the columns of $\mtrx{A}$ where $m \leq n$. Note that $\mtrx{AA^\mathsf{T}}$ is the dataset covariance matrix. Then a simple method is to randomly choose $k<m$ columns of $\mtrx{A}$ that form a matrix $\mtrx{B}$. Statistically, the SVD of $\mtrx{BB^\mathsf{T}}$ will be close to that of $\mtrx{AA^\mathsf{T}}$; thus it suffices to calculate the SVD of $\mtrx{B}$, the complexity of which, is only $O(k^2m)$.

% http://mathoverflow.net/questions/161252/what-is-the-time-complexity-of-truncated-svd


% subsection: Sparse matrix of details: Mozna uplne vyhodit, stejne to neni implementovano -- uvest jen v sekci Optimization nebo v Conclusion
