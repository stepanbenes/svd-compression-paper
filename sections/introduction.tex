\section{Introduction}
\label{sec:introduction}

% Introduction to field of FEM results post-processing.

Amount of data produced by a complex finite element analysis can be enormous and typical personal computer is not capable to store, process and visualize the results in reasonable time. Singular Value Decomposition (SVD) is a well known factorization method that provides rich information about matrix systems. One of its many applications is image compression where it can significantly reduce size of data representing image while preserving quality of image appearance.

The effort to reduce size of resulting data from complex finite element analyses to accelerate (or even make possible) post-processing using common personal computer is not new. In \cite{Benes2016}, there is one possible direction presented for research in this area -- the replacement of discrete data produced by finite element solver by continuous functions. These approximation functions can be then described by a small number of parameters. The main goal is to find the areas in domain where the output discrete function has predictable development and can be easily replaced by a simple continuous function, e.g. linear function. Although this approach has remarkable results for some classes of functions it is not applicable to a general problem. For some special cases, such as functions with discontinuities, the method has poor results because approximation error is too high and -- what is more important -- it cannot be determined in advance.

In this study it was decided to apply purely algebraic approach. Various methods were considered, e.g. discrete wavelet transform \cite{Lui2001} and discrete cosine transform \cite{Watson1994}, for their successful use in image compression. The reason why the SVD method was finally chosen is that it is well suited for all kinds of data and the implementation is straightforward, considering that the results from the FEM analyses can be viewed as a series of arbitrary rectangular matrices. Efficient data storage is connected with data structure \cite{Ivanyi2012, Ivanyi2014} and this area will be subject of further research.

Compression methods usually yield approximated data. In the following text, the term \textit{approximation error} denotes an error resulted from compression, i.e. difference between original results of FEM analysis and their compressed form. It should not be confused with the error of the Finite element method itself that yields approximate solution to mathematical problems used to model physical reality.

The quality of any compression method depends on the nature of input data. Therefore, several different non-trivial finite element analyses are used as benchmarks in this paper for an implementation of the compression algorithm. The ability to control the quality of compression is a key requirement for the implementation of the compression algorithm.

% summary of sections
The rest of the paper is organized as follows. Section \ref{sec:math} contains mathematical background of the SVD compression (i.e. SVD method itself, its use in low-rank matrix approximation, and description of the randomized SVD method). Implementation of the compression algorithm is presented in Section \ref{sec:implementation}. Section \ref{sec:results} summarizes the results of the benchmarks that were designed to measure quality of the output from the compression algorithm, and also the performance of its the implementation. The paper is concluded in Section \ref{sec:conclusion}.

% State of the art. Add references to work in the field of FEM results post-processing.
% Add references to related work and related usage of SVD for compression
% Description of all terms that are used in text?
